\chapter{Memory Management}

The \foenix\ comes with 4MB of RAM and 2MB of read-only flash
memory. In addition there is an additional 2MB of video RAM accessible by
Vicky.
The flash memory is used by the MCP kernel, or possibly other
operating systems.

Vicky can accesses memory at a much higher speed compared to the
processor and periodically needs to compose the video output. The
design prevents the processor from stealing cycles or do similar
tricks. To allow the processor to access video memory a FIFO (which is
a transition buffer) and DMA (which is a bulk transfer transaction)
are provided.

\notte{The C256U actually comes with 2MB of RAM while the C256U+
  variant has 4MB. Apart from that they are identical. The A2560U was
  also   sold like this, but so far all machines sold were in the 4MB
  configuration. In this guide the memory is normally said to be
  4MB. In reality 2MB is a lot of memory for a machine like this, so
  the difference does not really that much.}

\section*{Memory Models}

In general, imposing no restrictions for how much memory can be data
allows it to be arbitrary large and only restricted by the amount of
addressable memory. While making it simple to program such model, it
often results in inefficient use of the instruction set which means
your application grows larger and that it runs slower.

An example can be that there is an addressing mode that allow for fast
access to a restricted memory area, which typically can be 256 bytes
or 64K large, even if the actual memory is significantly larger,
e.g. 16MB.

In the C compiler this is expressed as a data model. A data model sets
rules for how to access memory and is often designed from the
instruction set perspective. The idea is to balance fast access to
memory combined with small size of the application. Depending of the
amount and structure of the needed data, you can select a model that
suits your needs. If the needs change during development, the compiler
can be configured to use a different data model to make the best use
of the capabilities of the processor. In essence, you can with a
simple command line setting cause the compiler to perform access to
memory using a potentially very different strategy. If you compare
this to writing your application in assembly language, you would need
to make changes potentially all over the place, with the compiler you
change a setting.

The compiler provides both code and data models suited to take full
advantage of the processor. These can be configured independently.

\subsection*{On the 68000}

The 68000 is fairly simple when it code and data models. There are two
code models provided and three data models.

As a rule of thumb, for small applications, you will normally use the
small code and small data models. As your program grows you may need
to switch to the large code model, but you can in most cases keep
using the small data model.

In most situations, the above is all you need to know. You probably
picked the 68000 to make your life easy, so enjoy skipping the rest of
this chapter and move on to the next one!

In certain situations, e.g. you need to interact with other
applicatons, or if you are curious to know what is exactly going on,
then you can continue below.

The small code model limits the code size to somewhere in the range
32K--64K and in rare situations a bit beyond that. It makes use of the
16-bit relative branches to jump between functions. If you get range
errors from the linker, you can solve it by switching to the large
code model which inserts absolute jumps that can reach any function
from any point. This will cause the code size of the applications to
grow a little bit and function calls will take a tiny amount of extra
time.

The small data model makes use of a base pointer in one of the address
registers. This register is named {\tt A4} and is permanently set
aside for this purpose. Static (global) variables are accessed based
on using this base pointer. This saves two bytes of application size
for each direct access to such variable and it executes faster
compared to the alternative absolute addressing. This limits the size
of all such static variables to 64K in the application.

Even though the small data model limites static data to 64K,
addressing memmory is still done using 32-bit pointers. This means
that you can have the small data model and make exceptions to it using
the {\tt \_\_far} address space keyword. Data allocated from the heap are
also also not affected by the 64K limitation.
As a result you can often keep using the small data model and get its
benefits even if your total data is significantly larger.

When do you not want to use the small data model you may ask. Well, in
general there are two situations. If you have very many small static
data objects it may be simpler to use the far-only data model. It may
also be easier to use the far-only model if you interact with other
applications and pass control back and forth. Such application will
most likely not be able to share the {\tt A4} base register with your
application. There are ways to mitigate this at interaction points,
but in general it may be easier to just avoid using the {\tt A4} base
pointer and get another data register for general use.

The final data model is the large data model, it is a mix between the
two in that it behaves as far-only but assumes that the {\tt A4} base
pointer is also set up in case use use the Near data attribute. It is
mostly provided for completeness and real world use-cases for it are
probably far between.

\subsection*{On the 65816}

There are several data models on the Calypsi C for 65816 and two code
models.

As a rule of thumb, for small applications that can reside inside the
first 64K (bank 0), use the small code and data models. As your
application grows you will probably want to put code outside of bank
0, select the large code model for this.

If you have more demands on data and want to place it outside bank 0
then the medium data model will take you a long way. It allocates static
(global) variables in a single bank (other than 0) called the Near
address space. 64K is actually a lot of memory for static data. Data pointers
are 24-bit (far) in this case so they can reach any location, but are
subject to not crossing 64K bank boundaries.

In case you have more static data than can fit in the single 64K Near
area, you can use the Far address space attribute on one or a
few large static objects.

The large and huge data models are mainly for very large applications
or when you want to avoid using {\tt \_\_far} or {\tt \_\_huge}
keywords completely. The size of the application grows additionally and
performance will suffer a bit, but it is an easy way to get large
amount of data into the memory space of the 65816.

I will not get deeper into the 65816 like was done for the 68000 above
as there are really a lot going on with the various data models on the
65816 and it is not as likely that you need as deep understanding as
there is less likely that you will need interopability with other code
as there not as much existing code and tools for this processor.

\subsection*{Address Spaces}

This section is if you want to understand address spaces and how they
related to data models more in depth. Feel free to skip this section
for now as you will most likely be able to do a lot of application
programming without understanding this in great detail. Should you
ever need this, you will most likely realize when it is time to dive
into this.

An /address space/ is a memory area that can be addressed in a
particular way. This can describe a specific limited memory area,
e.g. a single 64K area. It can also be larger area that is broken up
in segments, e.g. multiple 64K areas, the whole memory, video memory
or something else.

All address spaces has an associated memory attribute or keyword which
can be used to modify a pointer type to describe a particular address
space. How code is generated to access an address space is what the
data model is all about. While address spaces makes it possible to do
it, the data model is what makes it easy to use address spaces.

Thus, a data model is a set of rules for how to apply address spaces
by default in a given situation. One example can be that if we access
a static (absolute addressed) variable, we assume a certain address
space. Passing a pointer around may actually may mean it is described
by another address space, it can be the same address space or one that
is more general, perhaps even allowing the access to the full memory
range.

If this still sounds complicated, rest assured that the compiler knows
the rules and will assist to get it right by the means of either
implicit conversion rules that are applied when needed, or it will
request explicit conversions when that is required.

The selected data model is a way to describe the needs on memory your
application has and thereby minimize the amount of visible address
space attributes. The general idea is that you seldom should have to use
explicit address space attributes. If you find that you make use of many
such attributes, it is probably a sign that you should think about
switching to another data model.
